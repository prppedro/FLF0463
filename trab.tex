\documentclass[12pt,a4paper]{article}
% Trabalho de Política I
% Pedro T. R. Pinheiro (8983332)
\usepackage[utf8]{inputenc}
\usepackage[portuguese]{babel}
\usepackage{parskip}
\usepackage{setspace}
\usepackage[left=3.00cm, right=2.00cm, top=3.00cm, bottom=2.00cm]{geometry}
\usepackage{scrextend}

\newenvironment{citac}
{
	\begin{addmargin}
		[4cm]{1em} \footnotesize}{\normalfont 
	\end{addmargin}
	% Para exportar para RTF
%	\footnotesize }{ \normalfont
}

\author{Pedro T. R. Pinheiro\footnote{Aluno pelo departamento de filosofia 
da FFLCH/USP, de número USP 8983332. }}
\title{República e Liberdade Hoje\footnote{
	Trabalho desenvolvido para a matéria FLF0463 - Ética e Filosofia 
	Política III, ministrada pelo Prof. Dr. Alberto R. G. Barros. }}
\date{26 de junho de 2020}

\begin{document}
	\maketitle
	
	\onehalfspacing
	\setlength{\parskip}{0.5cm}

	A discussão acerca da liberdade não tem alicerces firmes. Falar sobre ela é, quase 
	sempre, pisar em solo arenoso, instável. Volátil em função das condições de 
	temperatura e pressão, é difícil dissociar os mais diversos conceitos de liberdade 
	de suas conjunturas culturais e sociais originais. Sobre o assunto, Isaiah Berlin, 
	importante pensador britânico, escrevera, na década de cinquenta, o que se tornaria 
	um dos mais importantes tratados dentro da filosofia política -- \textit{Dois 
	Conceitos de Liberdade}. A preocupação linguística da Berlin é latente: 

	\citac{É mero pedantismo confinar esta palavra [liberdade] [...] ou estamos, 
	como suspeito, a perigo de chamar qualquer melhora na situação social favorecida 
	por um ser humano como um aumento de sua liberdade; e não tornará isto o termo 
	tão vago a ponto de fazê-lo praticamente inútil? INSERIR FONTE INSERIR FONTE INSERIR FONTE} [PROCURAR TRECHO EM PT-BR]

	É sobre este ensaio denso, repleto de filigranas e discussões 
	importantes, com o qual debate-se, aqui, indiretamente, através do artigo de 
	Júlia César Casarin -- \textit{Isaiah Berlin: Afirmação e Limitação da Liberdade}. 
	Casarin faz um exame contextual acerca do trabalho de Berlin: 

	\citac{...}

	Portanto, Berlin parece conectar o debate iniciado por Benjamin Constant ao contexto 
	da política contemporânea. Um desafio hercúleo que Berlin parece admitir em seu 
	próprio texto, a cada vez em que nos lembra o quão escorregadias são as leis que 
	governam a política. 

	

	%\begin{thebibliography}{9}
	%	\bibitem{adorno}
	%	ADORNO, T. 
	%	\textit{Education after Auschwitz. }   
	%	\\\texttt{https://www.ime.usp.br/~tadeu/EDF0285/A10\_Adorno.pdf}
	%	
	%	
	%\end{thebibliography}
	
	\bibliographystyle{apalike}
	\bibliography{trab}

\end{document}
