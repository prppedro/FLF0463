\documentclass[12pt,a4paper]{article}
% Trabalho de Política I
% Pedro T. R. Pinheiro (8983332)
\usepackage[utf8]{inputenc}
\usepackage[portuguese]{babel}
\usepackage{parskip}
\usepackage{setspace}
\usepackage[left=3.00cm, right=2.00cm, top=3.00cm, bottom=2.00cm]{geometry}
\usepackage{scrextend}

\newenvironment{citac}
{
	\begin{addmargin}
		[4cm]{1em} \footnotesize}{\normalfont 
	\end{addmargin}
	% Para exportar para RTF
%	\footnotesize }{ \normalfont
}

\author{Pedro T. R. Pinheiro\footnote{Aluno pelo departamento de filosofia 
da FFLCH/USP, de número USP 8983332. }}
\title{República e Liberdade Hoje\footnote{
	Trabalho desenvolvido para a matéria FLF0463 - Ética e Filosofia 
	Política III, ministrada pelo Prof. Dr. Alberto R. G. Barros. }}
\date{26 de junho de 2020}

\begin{document}
	\maketitle
	
	\onehalfspacing
	\setlength{\parskip}{0.5cm}

	Liberdade. Um vocábulo, múltiplos sentidos. Dentre os milhares de anos 
	atravessados pela civilização ocidental -- e pela humanidade, como um 
	todo -- muitos brados por liberdade foram ouvidos. Se liberar, se 
	emancipar ou perseguir a perfeição intelectual? Não há, que se tenha 
	notícia, réplica perfeita a esta indagação, porquanto ela aí permanece. 
	No século XIX, Benjamin Constant comparara as liberdades conforme a 
	relativamente recente concepção iluminista às experimentadas por um 
	cidadão da Pólis na Antiguidade. Desde então, o mundo sofrera duas 
	revoluções industriais completas, passa, atualmente pela quarta e, 
	além disso, presenciou não apenas a edificação do muro de Berlin, como, 
	também, sua queda. Em tempos de uma proposta revolução industrial, 
	com o avanço da telecomunicação e do globalismo, é possível ainda que 
	ainda haja uma terceira e nova liberdade: a dos contemporâneos. 
	Uma tese forte, que, infelizmente, não há como ser discutida aqui 
	em detalhes. 

	Antes de se falar na liberdade em uma contemporaneidade tão revolta, 
	parece ser de bom tom revisitar a liberdade dos modernos em seu berço 
	liberal, grosso modo adotada como \textit{liberdade negativa} por 
	Isaiah Berlin, em seu famoso tratado ”Dois Conceitos de Liberdade”, 
	datado originalmente de 1958. Berlin reposiciona o debate acerca da 
	liberdade na cena da filosofia política, ao publicar este texto. 

	\textbf{[INSERIR E COMENTAR ALGUNS TRECHOS DE BERLIN, AQUI]}

	A tese de Berlin, entretanto, repousa sobre um paradigma cujo(a?)
	[FANCY STUFF] a faz ser bastante vulnerável às críticas de pensadores 
	com viés um pouco mais materialista, como os marxistas, por exemplo. 
	É o que se observa em um artigo de Júlio César Casarin [INSERT THE GODDAM NAME]
	
	[JUNTAR PARÁGRAFO]
	Nesse sentido, esta ”incomensurabilidade” professada pelo professor 
	de Oxford, parece ter um sentido ainda mais profundo: tão imenso  
	quanto o corte epistemológico entre os dogmáticos apontados por Kant 
	e os entusiastas da epistemologia forjada pela bigorna das ciências 
	naturais, há um abismo entre aqueles que postulam, de alguma maneira, 
	um aperfeiçoamento intelectual da sociedade e os que perseguem a ideia 
	de justiça social. [REPENSAR ISTO]
	

	%\begin{thebibliography}{9}
	%	\bibitem{adorno}
	%	ADORNO, T. 
	%	\textit{Education after Auschwitz. }   
	%	\\\texttt{https://www.ime.usp.br/~tadeu/EDF0285/A10\_Adorno.pdf}
	%	
	%	
	%\end{thebibliography}
	
	\bibliographystyle{apalike}
	\bibliography{trab}

\end{document}
