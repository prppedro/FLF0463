\documentclass[12pt,a4paper]{article}
% Trabalho de Política I
% Pedro T. R. Pinheiro (8983332)
\usepackage[utf8]{inputenc}
\usepackage[portuguese]{babel}
\usepackage{parskip}
\usepackage{setspace}
\usepackage[left=3.00cm, right=2.00cm, top=3.00cm, bottom=2.00cm]{geometry}
\usepackage{scrextend}

\newenvironment{citac}
{
	\begin{addmargin}
		[4cm]{1em} \footnotesize}{\normalfont 
	\end{addmargin}
	% Para exportar para RTF
%	\footnotesize }{ \normalfont
}

\author{Pedro T. R. Pinheiro\footnote{Aluno pelo departamento de filosofia 
da FFLCH/USP, de número USP 8983332. }}
\title{República e Liberdade Hoje\footnote{
	Trabalho desenvolvido para a matéria FLF0463 - Ética e Filosofia 
	Política III, ministrada pelo Prof. Dr. Alberto R. G. Barros. }}
\date{26 de junho de 2020}

\begin{document}
	\maketitle
	
	\onehalfspacing
	\setlength{\parskip}{0.5cm}

	É a \textit{liberdade} um dos mais escorregadios conceitos da 
	filosofia moral e política. ”Ser livre” é quase sempre uma 
	contingênia cujos contornos variam em função de tempo, cultura, 
	religião e preceitos filosóficos. De qualquer maneira, é a 
	liberdade componente integral às democracias contemporâneas. E, 
	como tal, profundamente influenciada pelo pensamento iluminista, 
	sob a égide de um nascente pensamento liberal. 

	É neste sentido que, para Benjamin Constant, difere a ”liberdade 
	dos antigos” da ”liberdade dos modernos”. Há, talvez pela primeira 
	vez na história, uma sólida defesa do indivíduo. Diferentemente do 
	homem livro grego, este tem um inato e inalienável direito à 
	liberdade. Grosso modo, pensar a liberdade como algo que se tem 
	a priori e de que se abre mão o mínimo possível constitui na 
	espinha dorsal do pensamento liberal -- a liberdade ”negativa”. 

	Contudo, esta liberdade ”positiva”, mais ou menos correspondente 
	à dos ”antigos” -- cujos moldes remontam às ideias de Sócrates na 
	\textit{República} de Platão -- ainda é bastante presente na 
	contemporaneidade, notavelmente carregada sob as asas do pensamento 
	político hegeliano, e, consequentemente, marxista. Talvez possa ser 
	melhor expressa como uma ”promessa de liberdade”, pois esta 
	emancipação ocorre como um objetivo em que todos os conflitos entre 
	os homens esteja sanado. 

	Como bem aponta Isaiah Berlin, estes modelos se colidem, entram em 
	choque. Choque este quase sempre observável em considerável parte 
	das discussões políticas de nosso período. Dentro deste contexto 
	parece estar, a primeira vista, o artigo ”Isaiah Berlin: Afirmação 
	e Limitação da Liberdade”, de J. Casarin, que aqui será usado como 
	forma de observar sob quais princípios e elementos fundamentais se 
	dá a crítica contemporânea ao pensamento liberal e a ideia de 
	liberdade negativa. 

	[FALTA CITAÇÃO E COMENTÁRIO]
	
	Casarin responde ao artigo ”Dois Conceitos de Liberdade”, escrito 
	por Berlin no final da década de 1950, em meio a um devastador 
	cenário pós-guerra e posterior à instauração de regimes autoritários 
	em diversas partes do mundo. Neste artigo, Berlin compara a liberdade 
	negativa à liberdade positiva aproximadamente nos mesmos moldes de 
	B. Constant. O filósofo pesa as duas concepções de liberdade, 
	examinando extensamente as consequências de cada uma delas, embora, 
	no fim das contas, acabe por optar por uma defesa da liberdade 
	negativa. 

	E é precisamente esta defesa um dos principais objetos de crítica 
	por parte de Casarin, que elenca três principais elementos da 
	crítica berliniana à liberdade positiva em contraste com a liberdade 
	negativa: 

	\begin{citac}
		Sua [Isaiah Berlin] defesa da primazia da liberdade negativa
		está assentada sobre três pilares principais: a afirmação 
		do pluralismo de valores, o argumento contra a “divisão do 
		eu” e, finalmente, um terceiro ponto, que recorre a 
		evidências empíricas e históricas e diz respeito à 
		possibilidade de a concepção positiva da liberdade degenerar 
		ela própria em um totalitarismo, avançando sobre os direitos 
		individuais e ameaçando a autonomia individual. 
		(\cite{casarin}, p. 285)
	\end{citac}

	O primeiro ponto versa acerca dos problemas com a visão pluralista de 
	Berlin. Para Casarin, a incomensurabilidade -- inerente a esta visão --
	conduziria a conflitos que tornariam, no limite, impossível a 
	coexistência entre, por exemplo, liberdade e justiça. E, mais do que 
	isso, ”por que motivo a liberdade individual é que deveria ter 
	prioridade às custas de outros valores igualmente relevantes?” 
	(\cite{casarin}, p. 287) Mais adiante, Casarin detalha sua objeção: 

	\begin{citac}
		Mas até que ponto podemos dar crédito a essa
		incomensurabilidade? Será verdade que valores
		arquetípicos não podem ser promovidos
		concomitante em um mesmo sistema político? A
		idéia de que liberdade e igualdade, ou liberdade
		individual e justiça, estão cada qual de um lado de
		uma gangorra, de modo que se uma delas ascende 
		necessariamente o fará às custas do descenso
		da outra, não nos parece apenas implausível, mas
		também um modelo primitivo demais para explicar 
		as complexas interações entre a liberdade, a 
		igualdade e a justiça social nas sociedades contemporâneas. 
		[...]
		Na era do capitalismo industrial em que Berlin escrevia, a
		autonomia individual não podia ser considerada 
		com seriedade e adequadamente promovida sem
		que se tivessem em conta questões de igualdade e
		de justiça, combinadas com a liberdade individual “negativa”.
		(\cite{casarin}, p. 287)
	\end{citac}

	Há, notavelmente, duas visões de mundo bastante díspares em debate. 
	Em certa medida, estas diferenças parecem surgir de visões 
	conflitantes acerca do que é um indivíduo. Para Berlin, o indivíduo 
	é, claramente, a unidade básica. Não existe ação coletiva sem que 
	haja, a priori, ação individual, ou, ainda antes, um pensamento 
	individual. Enquanto, para Casarin, a condição de indivíduo 
	emancipado depende, aparentemente, de certas condições políticas, 
	sociais e/ou econômicas. 

	Esta diferença basilar é visível em boa parte das objeções feitas 
	pelos defensores de sistemas baseados na liberdade positiva aos 
	liberais, e, com bastante frequência, cria debates circulares. A 
	este ponto, poder-se-ia argumentar que a perspectiva de Casarin 
	leva a uma liberdade individual garantida dentro de um contexto de 
	liberdade positiva, ao que Berlin, muito provavelmente, replicaria 
	mencionando os aspectos nefastos de se confiar num poder central 
	que garantirá esta prometida liberdade. Para um liberal, isto 
	simplesmente não faria sentido algum pois, a priori, não só é 
	esta liberdade parte do direito natural, como o estado é uma 
	concessão do indivíduo, e não o oposto. 

	Isto é muito claro para autores cujo contexto é o direito 
	consuetudinário, como é o caso de John Locke, por exemplo. 
	No modelo comunitário lockeano, é bem mais visível a parte 
	do indivíduo como agente constituinte da sociedade do que nos 
	estados contemporâneos, em que a centralização provoca, 
	frequentemente, uma distorção do indivíduo, perdido entre a 
	numerosa massa. 

	Por outro lado, pequenas comunidades não são capazes, sempre, de 
	autossustento, o que demanda uma crescente necessidade de intervenção 
	por parte da autoridade central, como se observa, por exemplo, nas 
	comunidades americanas. Desde a década de trinta, o governo americano 
	aumentou significativamente não apenas seu aparato militar, como, 
	também, o aparato social. E é notável a crescente demanda por 
	um estado mais presente. 

	Em suma, dentro de uma perspectiva liberal, o crescimento do aparato 
	estatal eclipsa o indivíduo. E, dentro de uma perspectiva, por assim 
	dizer, ”progressista”, o crescimento do indivíduo aprofunda as feridas 
	da desiguldade social. São perspectivas irreconciliáveis, sob as quais 
	Berlin pode estar basnte correto ao falar sobre incomensurabilidade. 

	O segundo ponto de Casarin -- sobre a ”divisão do eu” e o 
	antipaternalismo -- [...] 

	[...]
	\newpage

	A tese de Berlin, entretanto, repousa sobre um paradigma cujo(a?)
	[FANCY STUFF] a faz ser bastante vulnerável às críticas de pensadores 
	com viés um pouco mais materialista, como os marxistas, por exemplo. 
	É o que se observa em um artigo de Júlio César Casarin [INSERT THE GODDAM NAME]
	
	[JUNTAR PARÁGRAFO]
	Nesse sentido, esta ”incomensurabilidade” professada pelo professor 
	de Oxford, parece ter um sentido ainda mais profundo: tão imenso  
	quanto o corte epistemológico entre os dogmáticos apontados por Kant 
	e os entusiastas da epistemologia forjada pela bigorna das ciências 
	naturais, há um abismo entre aqueles que postulam, de alguma maneira, 
	um aperfeiçoamento intelectual da sociedade e os que perseguem a ideia 
	de justiça social. [REPENSAR ISTO]
	

	%\begin{thebibliography}{9}
	%	\bibitem{adorno}
	%	ADORNO, T. 
	%	\textit{Education after Auschwitz. }   
	%	\\\texttt{https://www.ime.usp.br/~tadeu/EDF0285/A10\_Adorno.pdf}
	%	
	%	
	%\end{thebibliography}
	
	\bibliographystyle{apalike}
	\bibliography{trab}

\end{document}
