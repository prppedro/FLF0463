\documentclass[12pt,a4paper]{article}
% Trabalho de Política I
% Pedro T. R. Pinheiro (8983332)
\usepackage[utf8]{inputenc}
\usepackage[portuguese]{babel}
\usepackage{parskip}
\usepackage{setspace}
\usepackage[left=3.00cm, right=2.00cm, top=3.00cm, bottom=2.00cm]{geometry}
\usepackage{scrextend}

\newenvironment{citac}
{
	\begin{addmargin}
		[4cm]{1em} \footnotesize}{\normalfont 
	\end{addmargin}
	% Para exportar para RTF
%	\footnotesize }{ \normalfont
}

\author{Pedro T. R. Pinheiro\footnote{Aluno pelo departamento de filosofia 
da FFLCH/USP, de número USP 8983332. }}
\title{República e Liberdade Hoje\footnote{
	Trabalho desenvolvido para a matéria FLF0463 - Ética e Filosofia 
	Política III, ministrada pelo Prof. Dr. Alberto R. G. Barros. }}
\date{26 de junho de 2020}

\begin{document}
	\maketitle
	
	\onehalfspacing
	\setlength{\parskip}{0.5cm}

	É a \textit{liberdade} um dos mais escorregadios conceitos da 
	filosofia moral e política. ”Ser livre” é quase sempre uma 
	contingênia cujos contornos variam em função de tempo, cultura, 
	religião e preceitos filosóficos. De qualquer maneira, é a 
	liberdade componente integral às democracias contemporâneas. E, 
	como tal, profundamente influenciada pelo pensamento iluminista, 
	sob a égide de um nascente pensamento liberal. 

	É neste sentido que, para Benjamin Constant, difere a ”liberdade 
	dos antigos” da ”liberdade dos modernos”. Há, talvez pela primeira 
	vez na história, uma sólida defesa do indivíduo. Diferentemente do 
	homem livro grego, este tem um inato e inalienável direito à 
	liberdade. Grosso modo, pensar a liberdade como algo que se tem 
	a priori e de que se abre mão o mínimo possível constitui na 
	espinha dorsal do pensamento liberal -- a liberdade ”negativa”. 

	Contudo, esta liberdade ”positiva”, mais ou menos correspondente 
	à dos ”antigos” -- cujos moldes remontam às ideias de Sócrates na 
	\textit{República} de Platão -- ainda é bastante presente na 
	contemporaneidade, notavelmente carregada sob as asas do pensamento 
	político hegeliano, e, consequentemente, marxista. Talvez possa ser 
	melhor expressa como uma ”promessa de liberdade”, pois esta 
	emancipação ocorre como um objetivo em que todos os conflitos entre 
	os homens esteja sanado. 

	Como bem aponta Isaiah Berlin, estes modelos se colidem, entram em 
	choque. Choque este quase sempre observável em considerável parte 
	das discussões políticas de nosso período. 

	[...]
	


	Antes de se falar na liberdade em uma contemporaneidade tão revolta, 
	parece ser de bom tom revisitar a liberdade dos modernos em seu berço 
	liberal, grosso modo adotada como \textit{liberdade negativa} por 
	Isaiah Berlin, em seu famoso tratado ”Dois Conceitos de Liberdade”, 
	datado originalmente de 1958. Berlin reposiciona o debate acerca da 
	liberdade na cena da filosofia política, ao publicar este texto. 

	\textbf{[INSERIR E COMENTAR ALGUNS TRECHOS DE BERLIN, AQUI]}

	A tese de Berlin, entretanto, repousa sobre um paradigma cujo(a?)
	[FANCY STUFF] a faz ser bastante vulnerável às críticas de pensadores 
	com viés um pouco mais materialista, como os marxistas, por exemplo. 
	É o que se observa em um artigo de Júlio César Casarin [INSERT THE GODDAM NAME]
	
	[JUNTAR PARÁGRAFO]
	Nesse sentido, esta ”incomensurabilidade” professada pelo professor 
	de Oxford, parece ter um sentido ainda mais profundo: tão imenso  
	quanto o corte epistemológico entre os dogmáticos apontados por Kant 
	e os entusiastas da epistemologia forjada pela bigorna das ciências 
	naturais, há um abismo entre aqueles que postulam, de alguma maneira, 
	um aperfeiçoamento intelectual da sociedade e os que perseguem a ideia 
	de justiça social. [REPENSAR ISTO]
	

	%\begin{thebibliography}{9}
	%	\bibitem{adorno}
	%	ADORNO, T. 
	%	\textit{Education after Auschwitz. }   
	%	\\\texttt{https://www.ime.usp.br/~tadeu/EDF0285/A10\_Adorno.pdf}
	%	
	%	
	%\end{thebibliography}
	
	\bibliographystyle{apalike}
	\bibliography{trab}

\end{document}
